%
% statsnotes.tex:
% Notes on the statistics in the survey.
%
% Copyright (c) 2003 Chris Lightfoot. All rights reserved.
% Email: chris@ex-parrot.com; WWW: http://www.ex-parrot.com/~chris/
%
% $Id: statsnotes.tex,v 1.2 2003/07/16 22:18:02 chris Exp $
%
\documentclass[a4paper,12pt]{article}

\usepackage{mathpple}
\usepackage{hyperref}
\usepackage{url}

\newcommand\cov{{\rm cov}}
\newcommand\s{{\bf s}}
\newcommand\Ss{{\bf ss}}

\title{Notes on methods for the political survey}
\author{Chris Lightfoot, $\langle$chris@ex-parrot.com$\rangle$}

\begin{document}

\maketitle

\abstract{
Some brief notes summarising the statistics and method in use.
}

\section{Definitions}

\begin{tabular}{rl}
$x_i$           &   $i$th variate drawn from distribution $X$           \\
$N$             &   number of observations $x_i$                        \\
$\s_x$          &   sum of observations of $x_i$, $\sum_i^N x_i$        \\
$\s_{xx}$       &   sum of observations of $x_i^2$, $\sum_i^N x_i^2$    \\
$\bar{x}$       &   mean of the $x_i$, $\s_x / N$                       \\
$\cov(x, y)$    &   covariance between $X$ and $Y$, 
                $\overline{xy}-\bar{x}\bar{y} = \s_{xy}/N - \s_x \s_y / N^2$\\
$\Ss_x$         &   $\s_x - \bar{x}$                                        \\
$a + b x$       &   best fit line for $y$ as a function of $x$;
                    $b = \Ss_{xy} / \Ss_{xx}$, $a = \bar{y} - b \bar{x}$    \\
$r^2$           &   correlation coefficient for linear fit;
                    $r^2 = \Ss_{xy}^2 / \Ss_{xx} \Ss_{yy}$                  \\
$C_{mn}$        &   covariance between variables $m$ and $n$; element in
                    covariance matrix $C$
\end{tabular}


\section{Statements and answers}

Each {\em statement} or proposition has a `normal' and a `converse' form. These
are supposed to be antonyms, for instance

\begin{tabular}{rl}
{\bf normal }   & Family is more important than society.        \\
{\bf converse } & Society is more important than family.        \\
\end{tabular}

... though many are more ambiguous than that.

Each {\em answer} is an integer between $-2$ and $+2$ inclusive; we assign these
the following labels:

\begin{tabular}{rl}
$-2$    &       disagree strongly   \\
$-1$    &       disagree            \\
$ 0$    &       no opinion          \\
$+1$    &       agree               \\
$+2$    &       agree strongly      \\
\end{tabular}

Each respondent is presented with a random mixture of normal and converse
statements; the statements are presented in random order. A few statements are
repeated in both forms.

When we put the same statement to a respondent in both normal and converse
forms, we record both answers, $x$ and $y$. We then compute a best-fit line
$y = f(x) = a + bx$ between each pair of answers for that statement, and
measure the goodness-of-fit $r^2$. For the remaining analysis we use either
answers to the statement in its normal form or, if they are not available,
answers in the converse form mapped to the normal form through $f(x)$. We use
$r^2$ to verify that the two forms of the statement are fair antonyms.

\section{Principal component analysis}

Once we have a large number of responses to all the statements, we can compute
the covariance between the $m$th and $n$th statements,
$\cov(x_m, x_n) = C_{mn}$ for each $m$, $n$. $C$ is called the covariance
matrix. Its eigenvectors ${\bf e}_k$ define {\em principal axes}, linear
combinations of the various statements which describe the directions of maximum
variation in the data. The corresponding eigenvalues $\lambda_k$ tell us how
significant each axis of variation is, the eigenvectors with the largest
$\lambda_k$ being the most significant.

\section{References}

\begin{itemize}
\item{Covariance, \url{http://mathworld.wolfram.com/Covariance.html}}
\item{Linear least squares fitting, \url{http://mathworld.wolfram.com/LeastSquaresFitting.html}}
\item{Principal component analysis, \url{http://www.cis.hut.fi/~jhollmen/dippa/node30.html}}
\end{itemize}

\end{document}
